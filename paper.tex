\documentclass{article}
\usepackage[round]{natbib}
\usepackage{amsmath,amssymb,amsfonts}%
\usepackage{geometry}%
\usepackage{color}
\usepackage{graphicx}
\usepackage{authblk}
\usepackage{nameref}
\usepackage[right]{lineno}

\begin{document}

\linenumbers
\title{What is an Ancestral Recombination Graph?}
% First authors
\author{Author McAuthory}
% Corresponding

\maketitle

% JK: this is a rough first pass for a slightly different paper. Needs
% substantial revision.
\begin{abstract}
It has recently become possible to infer genetic ancestry in the presence of
recombination at scale for the first time. This has created exciting
possibilities, with many downstream applications becoming possible based on
these inferred genealogies. The inferred genetic ancestries are usually
referred to as Ancestral Recombination Graphs, or ARGs. Although initially well
defined as a graphical representation of the coalescent with recombination
stochastic process, the interpretation has become unclear as it is now
also understood to represent a particular realisation of a genetic
ancestry.
Inference methods do not all infer the same information:
although all output marginal trees along the genome, there is a great deal of
variation in how much information is inferred and retained in the relationships
between the trees. An important programme of work over the coming years is to
develop and refine ancestry inference methods, assessing the relative strengths
and weaknesses of the various approaches. Using the blanket term ARG hides the
important differences in the approaches, and hampers our ability to compare and
improve inference methods. In this paper we discuss the different approaches we
may use to encode genetic ancestry, and of the differing amounts of information
about the ancestral process that we can represent and hope to observe. We
classify existing inference methods according to the type of ancestry they
infer, and discuss the strengths and weaknesses of the inferred structure for
downstream applications.
\end{abstract}

\textbf{Keywords:} Ancestral Recombination Graphs

\section*{Introduction}

\section*{The ARG as a stochastic process}
The coalescent
process~\citep{kingman1982coalescent,kingman1982genealogy,hudson1983testing,
tajima1983evolutionary}

The Ancestral Recombination Graph (ARG) was introduced by
Griffiths~\citep{griffiths1991two,griffiths1997ancestral}.
The ``big'' ARG~\citep{ethier1990two},
and the ``little ARG'' traversed by
Hudson's algorithm~\citep{hudson1983properties}.

\section*{The ARG as an encoding of genetic ancestry}

\citep[e.g.][]{minichiello2006mapping,mathieson2020ancestry}.


\bibliographystyle{plainnat}
\bibliography{paper}

\end{document}
