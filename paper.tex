\documentclass{article}
\usepackage[round]{natbib}
\usepackage{amsmath,amssymb,amsfonts}%
\usepackage{geometry}%
\usepackage{color}
\usepackage{graphicx}
\usepackage{authblk}
\usepackage{nameref}
\usepackage[right]{lineno}

\begin{document}

\linenumbers
\title{What is an Ancestral Recombination Graph?}
% First authors
\author{Authory McAuthor}
% Corresponding

\maketitle

\begin{abstract}
The ARG. Argh!
\end{abstract}

\textbf{Keywords:} Ancestral Recombination Graphs

\section*{Introduction}


\section*{The ARG as a stochastic process}
The coalescent
process~\citep{kingman1982coalescent,kingman1982genealogy,hudson1983testing,
tajima1983evolutionary}

The Ancestral Recombination Graph (ARG) was introduced by
Griffiths~\citep{griffiths1991two,griffiths1997ancestral}.
The ``big'' ARG~\citep{ethier1990two},
and the ``little ARG'' traversed by
Hudson's algorithm~\citep{hudson1983properties}.

\section*{The ARG as an encoding of genetic ancestry}

\citep[e.g.][]{minichiello2006mapping,mathieson2020ancestry}.


\bibliographystyle{plainnat}
\bibliography{paper}

\end{document}
